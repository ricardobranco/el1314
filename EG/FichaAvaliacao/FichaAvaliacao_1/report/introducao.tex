\section{Introdução}

A gramática independente de contexto, GIC, abaixo apresentada, define uma linguagem específica para descrever os livros e CDs disponíveis numa Biblioteca e o estado associado a cada um (livre ou emprestado), à semelhança do que acontece nas bibliotecas da UM.
O Símbolo Inicial é Biblioteca, os Símbolos Terminais são escritos em maiúsculas (pseudo-terminais) ou em maiúscula entre apostrofes (palavras-reservadas e sinais de pontuação), e a string nula é denotada por \&; os restantes (sempre em minúsculas) serão os Símbolos Não-Terminais.
Neste contexto e após analisar a GIC dada, torna-se necessário responder a algumas questões.
\begin{verbatim}
    p0: Biblioteca --> Registos
    p1: Registos --> Registo
    p2:     | Registos ',' Registo
    p3: Registo --> '[' REGISTO Descricao EXISTENCIAS Existencias ']'
    p4: Descricao --> Referencia ':' Catalogo '-' Tipo '-' Titulo '(' Autor ')'
    Editora '-' Ano
    P5: Referencia --> id
    p6: Tipo --> LIVRO
    p7:     | CDROM
    p8:     | OUTRO
    p9: Titulo --> string
    p10: Autor --> string
    p11: Editora --> string
    p12: Ano --> num
    p13: Catalogo --> BGUM
    p14:     | BAUM
    p15:     | OUTRO
    p19: Existencias --> Estado
    p20:     | Existencias ',' Estado
    p21: Estado --> CodigoBarras Disponib
    p22: CodigoBarras --> id
    p23: Disponib --> ESTANTE
    p24:     | PERMANENTE
    p25:     | EMPRESTADO DataDev
    p26: DataDev --> Ano '-' Mes '-' Dia
    p27: Mes --> num
    p28: Dia --> num

\end{verbatim}


a) Escreva uma frase válida da linguagem gerada pela GIC dada, mostrando a respectiva Árvore de Derivação.

b) Altere a gramáatica de modo a permitir que cada livro tenha mais de um Autor.

c) O par de produções p1/p2 define uma lista com recursividade à esquerda. Altere esse par para usarrecursividade à direita e mostre, através das árvores de derivação, a diferença entre ambos os esquemas iterativos.

d) Escreva as funções de um parser RD-puro (recursivo-descendente) para reconhecer o Símbolo Existencias e seus derivados, isto é, os símbolos que apareçam no lado direito das produções.

e) Construa o estado inicial do autómato LR(0) para gramática dada e os estados que dele saem.

f) No sentido de começar a avaliar a qualidade desta gramática calcule algumas métricas de tamanho
sobre ela: número de símbolos de cada tipo; número de produções; número de produções unitárias;
comprimento médio e máximo dos lados direitos das produções; ...

g) Transforme a GIC dada numa gramática de atributos, GA, reconhecível pelo AnTLR, para:
\begin{itemize}
\item calcular e imprimir: o número de registos; e o número de livros existentes para cada registo.
\item calcular o número total de livros com estado PERMANENTE.
\item identificar e listar por ordem alfabética os títulos dos livros.
\item verificar se não existem registos com a mesma referência.
\end{itemize}

\newpage
